%% nada
\documentclass[9pt,technote]{IEEEtran}

\usepackage[nomarkers]{endfloat}

\usepackage[cmex10]{amsmath}
\interdisplaylinepenalty=2500

\ifCLASSOPTIONcompsoc
\usepackage[caption=false,font=normalsize,labelfont=sf,textfont=sf]{subfig}
\else
\usepackage[caption=false,font=footnotesize]{subfig}
\fi

\begin{document}

\author{\IEEEauthorblockN{Michael Shell}
\IEEEauthorblockA{School of Electrical and\\
Computer Engineering\\
Georgia Institute of Technology\\
Atlanta, Georgia 30332--0250\\
Email: mshell@ece.gatech.edu}
\and
\IEEEauthorblockN{Homer Simpson}
\IEEEauthorblockA{Twentieth Century Fox\\
Springfield, USA\\
Email: homer@thesimpsons.com}
\and
\IEEEauthorblockN{James Kirk\\
and Montgomery Scott}
\IEEEauthorblockA{Starfleet Academy\\
San Francisco, California 96678-2391\\
Telephone: (800) 555--1212\\
Fax: (888) 555--1212}}


\author{\IEEEauthorblockN{Michael Shell\IEEEauthorrefmark{1}, Homer Simpson\IEEEauthorrefmark{2}, James Kirk\IEEEauthorrefmark{3}, Montgomery Scott\IEEEauthorrefmark{3} and Eldon Tyrell\IEEEauthorrefmark{4}}
\IEEEauthorblockA{\IEEEauthorrefmark{1}School of Electrical and Computer Engineering\\
Georgia Institute of Technology, Atlanta, Georgia 30
332--0250\\
Email: mshell@ece.gatech.edu}
\IEEEauthorblockA{\IEEEauthorrefmark{2}Twentieth Century Fox, Springfield, USA\\
Email: homer@thesimpsons.com}
\IEEEauthorblockA{\IEEEauthorrefmark{3}Starfleet Academy, San Francisco, California 96678-2391\\
Telephone: (800) 555--1212, Fax: (888) 555--1212}
\IEEEauthorblockA{\IEEEauthorrefmark{4}Tyrell Inc.,123 Replicant Street, Los Angeles, California 90210--4321}}


\markboth{Journal of Quantum Telecommunications, ~Vol. ~1, No. ~1, ~January ~2025}{Shell \MakeLowercase{\textit{et al.}}: A Novel Tin Can Link}


\IEEEpubid{0000--0000/00\$00.00 ~\copyright ~2014 IEEE}


\IEEEtitleabstractindextext{
\begin{abstract}
We propose ...
\end{abstract}
\begin{IEEEkeywords}
Broad band networks, quality of service, WDM.
\end{IEEEkeywords}}


\begin{equation}
\label{eqn_example}
x = \sum\limits_{i=0}^{z} 2^{i}Q
\end{equation}
% ... as can be seen in (\ref{eqn_example}).



\setlength{\arraycolsep}{0.0em}
\begin{eqnarray}
Z&{}={}&x_1 + x_2 + x_3 + x_4 + x_5 + x_6\nonumber\\
&&+a + b\\
&&+{}a + b\\
&&{}+a + b\\
&&{+}\:a + b
\end{eqnarray}
\setlength{\arraycolsep}{5pt}



\begin{table}[!t]
\renewcommand{\arraystretch}{1.3}
\caption{A Simple Example Table}
\label{table_example}
\centering
\begin{tabular}{c||c}
\hline
\bfseries First & \bfseries Next\\
\hline\hline
1.0 & 2.0\\
\hline
\end{tabular}
\end{table}


\newcounter{MYtempeqncnt}

\begin{figure*}[!t]
% ensure that we have normalsize text
\normalsize
% Store the current equation number.
\setcounter{MYtempeqncnt}{\value{equation}}
% Set the equation number to one less than the one
% desired for the first equation here.
% The value here will have to changed if equations
% are added or removed prior to the place these
% equations are referenced in the main text.
\setcounter{equation}{5}
\begin{equation}
\label{eqn_dbl_x}
x = 5 + 7 + 9 + 11 + 13 + 15 + 17 + 19 + 21+ 23 + 25
+ 27 + 29 + 31
\end{equation}
\begin{equation}
\label{eqn_dbl_y}
y = 4 + 6 + 8 + 10 + 12 + 14 + 16 + 18 + 20+ 22 + 24
+ 26 + 28 + 30
\end{equation}
% Restore the current equation number.
\setcounter{equation}{\value{MYtempeqncnt}}
% IEEE uses as a separator
\hrulefill
% The spacer can be tweaked to stop underfull vboxes.
\vspace*{4pt}
\end{figure*}

% % The previous equation was number five.
% % Account for the double column equations here.
% \addtocounter{equation}{2}
% As can be seen in (\ref{eqn_dbl_x}) and
% (\ref{eqn_dbl_y}) at the top of the page ...


\begin{description}[\IEEEsetlabelwidth{$\alpha\omega\pi\theta\mu$}\IEEEusemathlabelsep]
\item[$\gamma\delta\beta$] Is the index of..
\item[$\alpha\omega\pi\theta\mu$] Gives the..
\end{description}



\begin{enumerate}[\IEEEsetlabelwidth{2)}]
\item blah
\item blah
% 2 items total
\end{enumerate}



\newtheorem{theorem}{Theorem}
\begin{theorem}[Einstein-Podolsky-Rosenberg]
Nada
\end{theorem}
\begin{IEEEproof}
Nada
\end{IEEEproof}


\begin{IEEEeqnarray}{rCl}
Z&=&x_1 + x_2 + x_3 + x_4 + x_5 + x_6\IEEEnonumber\\
&&+\:a + b%
\end{IEEEeqnarray}

\begin{IEEEeqnarray}{Rl}
Z=&x_1 + x_2 + x_3 + x_4 + x_5 + x_6\IEEEnonumber\\
&+\:a + b%
\end{IEEEeqnarray}


\begin{IEEEeqnarray}{rCl}
A_1&=&7\IEEEyesnumber\IEEEyessubnumber\\
A_2&=&b+1\IEEEyessubnumber\\
\noalign{\noindent and\vspace{\jot}}A_3&=&d+2\IEEEyessubnumber%
\end{IEEEeqnarray}


\begin{IEEEeqnarray}[\setlength{\nulldelimiterspace}{0pt}]{rl,s}&x,&for $x \geq 0$\IEEEyesnumber\IEEEyessubnumber\\*
[-0.625\normalbaselineskip]
\smash{|x|=\left\{\IEEEstrut[3\jot][3\jot]\right.}&&
\nonumber\\*[-0.625\normalbaselineskip]
&-x,&for $x < 0$\IEEEyessubnumber
\end{IEEEeqnarray}

\begin{equation}
I = \left(\begin{IEEEeqnarraybox*}[][c]{,c/c/c,}
1&0&0\\
0&1&0\\
0&0&1%
\end{IEEEeqnarraybox*}\right)
\end{equation}


\begin{table}[!t]
\centering
\caption{Network Delay as a Function of Load}
\label{table_delay}
\begin{IEEEeqnarraybox}[\IEEEeqnarraystrutmode\IEEEeqnarraystrutsizeadd{2pt}{0pt}]{x/r/Vx/r/v/r/x}
\IEEEeqnarraydblrulerowcut\\
&&&&\IEEEeqnarraymulticol{3}{t}{Average Delay}&\\
&\hfill\raisebox{-3pt}[0pt][0pt]{$\beta$}\hfill&&\IEEEeqnarraymulticol{5}{h}{}%
\IEEEeqnarraystrutsize{0pt}{0pt}\\
&&&&\hfill\lambda_{\mbox{min}}\hfill&&\hfill\lambda_{\mbox{max\vphantom{i}}}\hfill&\IEEEeqnarraystrutsizeadd{0pt}{2pt}\\
\IEEEeqnarraydblrulerowcut\\
&1&&&0.057&& 0.172&\\
&10&&& 0.124&& 0.536&\\
&100&&& 0.830&& 0.905\rlap{\textsuperscript{*}}&\\
\IEEEeqnarraydblrulerowcut\\
&\IEEEeqnarraymulticol{7}{s}{\scriptsize\textsuperscript{*}limited usability}%
\end{IEEEeqnarraybox}
\end{table}



\section{Proof of the First Zonklar Equation}
\appendix[Proof of the Zonklar Equations]


\section*{Acknowledgment}
\addcontentsline{toc}{section}{Acknowledgment}


\begin{IEEEbiography}[Hola]{MichaelShell}
\end{IEEEbiography}


\IEEEraisesectionheading{\section{Introduction}\label{sec:introduction}}

\end{document}  %End of document.
