%% nada
\documentclass[9pt,technote]{IEEEtran}

\usepackage[nomarkers]{endfloat}

\usepackage[cmex10]{amsmath}
\interdisplaylinepenalty=2500

\ifCLASSOPTIONcompsoc
\usepackage[caption=false,font=normalsize,labelfon
t=sf,textfont=sf]{subfig}
\else
\usepackage[caption=false,font=footnotesize]{subfig}
\fi

\begin{document}





\begin{IEEEeqnarray}{rCl}
Z&=&x_1 + x_2 + x_3 + x_4 + x_5 + x_6\IEEEnonumber\\
&&+\:a + b%
\end{IEEEeqnarray}

\begin{IEEEeqnarray}{Rl}
Z=&x_1 + x_2 + x_3 + x_4 + x_5 + x_6\IEEEnonumber\\
&+\:a + b%
\end{IEEEeqnarray}


\begin{IEEEeqnarray}{rCl}
A_1&=&7\IEEEyesnumber\IEEEyessubnumber\\
A_2&=&b+1\IEEEyessubnumber\\
\noalign{\noindent and\vspace{\jot}}A_3&=&d+2\IEEEyessubnumber%
\end{IEEEeqnarray}


\begin{IEEEeqnarray}[\setlength{\nulldelimiterspace}{0pt}]{rl,s}&x,&for $x \geq 0$\IEEEyesnumber\IEEEyessubnumber\\*
[-0.625\normalbaselineskip]
\smash{|x|=\left\{\IEEEstrut[3\jot][3\jot]\right.}&&
\nonumber\\*[-0.625\normalbaselineskip]
&-x,&for $x < 0$\IEEEyessubnumber
\end{IEEEeqnarray}


\begin{equation}
\label{eqn_example}
x = \sum\limits_{i=0}^{z} 2^{i}Q
\end{equation}
% ... as can be seen in (\ref{eqn_example}).


\setlength{\arraycolsep}{0.0em}
\begin{eqnarray}
Z&{}={}&x_1 + x_2 + x_3 + x_4 + x_5 + x_6\nonumber\\
&&+a + b\\
&&+{}a + b\\
&&{}+a + b\\
&&{+}\:a + b
\end{eqnarray}
\setlength{\arraycolsep}{5pt}


\begin{equation}
I = \left(\begin{IEEEeqnarraybox*}[][c]{,c/c/c,}
1&0&0\\
0&1&0\\
0&0&1%
\end{IEEEeqnarraybox*}\right)
\end{equation}


\newcounter{MYtempeqncnt}
\begin{figure*}[!t]
	\setcounter{MYtempeqncnt}{\value{equation}}
	\setcounter{equation}{5}
	\begin{equation}
		\label{eqn_dbl_x}
		x = 5 + 7 + 9 + 11 + 13 + 15 + 17 + 19 + 21+ 23 + 25
		+ 27 + 29 + 31
	\end{equation}
	\begin{equation}
		\label{eqn_dbl_y}
		y = 4 + 6 + 8 + 10 + 12 + 14 + 16 + 18 + 20+ 22 + 24
		+ 26 + 28 + 30
	\end{equation}
	% Restore the current equation number.
	\setcounter{equation}{\value{MYtempeqncnt}}
\end{figure*}


\end{document}  %End of document.
